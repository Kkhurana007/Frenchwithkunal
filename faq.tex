% Options for packages loaded elsewhere
\PassOptionsToPackage{unicode}{hyperref}
\PassOptionsToPackage{hyphens}{url}
\PassOptionsToPackage{dvipsnames,svgnames,x11names}{xcolor}
%
\documentclass[
  16pt,
  letterpaper,
  DIV=11,
  numbers=noendperiod]{scrartcl}

\usepackage{amsmath,amssymb}
\usepackage{iftex}
\ifPDFTeX
  \usepackage[T1]{fontenc}
  \usepackage[utf8]{inputenc}
  \usepackage{textcomp} % provide euro and other symbols
\else % if luatex or xetex
  \usepackage{unicode-math}
  \defaultfontfeatures{Scale=MatchLowercase}
  \defaultfontfeatures[\rmfamily]{Ligatures=TeX,Scale=1}
\fi
\usepackage{lmodern}
\ifPDFTeX\else  
    % xetex/luatex font selection
\fi
% Use upquote if available, for straight quotes in verbatim environments
\IfFileExists{upquote.sty}{\usepackage{upquote}}{}
\IfFileExists{microtype.sty}{% use microtype if available
  \usepackage[]{microtype}
  \UseMicrotypeSet[protrusion]{basicmath} % disable protrusion for tt fonts
}{}
\makeatletter
\@ifundefined{KOMAClassName}{% if non-KOMA class
  \IfFileExists{parskip.sty}{%
    \usepackage{parskip}
  }{% else
    \setlength{\parindent}{0pt}
    \setlength{\parskip}{6pt plus 2pt minus 1pt}}
}{% if KOMA class
  \KOMAoptions{parskip=half}}
\makeatother
\usepackage{xcolor}
\setlength{\emergencystretch}{3em} % prevent overfull lines
\setcounter{secnumdepth}{-\maxdimen} % remove section numbering
% Make \paragraph and \subparagraph free-standing
\makeatletter
\ifx\paragraph\undefined\else
  \let\oldparagraph\paragraph
  \renewcommand{\paragraph}{
    \@ifstar
      \xxxParagraphStar
      \xxxParagraphNoStar
  }
  \newcommand{\xxxParagraphStar}[1]{\oldparagraph*{#1}\mbox{}}
  \newcommand{\xxxParagraphNoStar}[1]{\oldparagraph{#1}\mbox{}}
\fi
\ifx\subparagraph\undefined\else
  \let\oldsubparagraph\subparagraph
  \renewcommand{\subparagraph}{
    \@ifstar
      \xxxSubParagraphStar
      \xxxSubParagraphNoStar
  }
  \newcommand{\xxxSubParagraphStar}[1]{\oldsubparagraph*{#1}\mbox{}}
  \newcommand{\xxxSubParagraphNoStar}[1]{\oldsubparagraph{#1}\mbox{}}
\fi
\makeatother


\providecommand{\tightlist}{%
  \setlength{\itemsep}{0pt}\setlength{\parskip}{0pt}}\usepackage{longtable,booktabs,array}
\usepackage{calc} % for calculating minipage widths
% Correct order of tables after \paragraph or \subparagraph
\usepackage{etoolbox}
\makeatletter
\patchcmd\longtable{\par}{\if@noskipsec\mbox{}\fi\par}{}{}
\makeatother
% Allow footnotes in longtable head/foot
\IfFileExists{footnotehyper.sty}{\usepackage{footnotehyper}}{\usepackage{footnote}}
\makesavenoteenv{longtable}
\usepackage{graphicx}
\makeatletter
\newsavebox\pandoc@box
\newcommand*\pandocbounded[1]{% scales image to fit in text height/width
  \sbox\pandoc@box{#1}%
  \Gscale@div\@tempa{\textheight}{\dimexpr\ht\pandoc@box+\dp\pandoc@box\relax}%
  \Gscale@div\@tempb{\linewidth}{\wd\pandoc@box}%
  \ifdim\@tempb\p@<\@tempa\p@\let\@tempa\@tempb\fi% select the smaller of both
  \ifdim\@tempa\p@<\p@\scalebox{\@tempa}{\usebox\pandoc@box}%
  \else\usebox{\pandoc@box}%
  \fi%
}
% Set default figure placement to htbp
\def\fps@figure{htbp}
\makeatother

\KOMAoption{captions}{tableheading}
\makeatletter
\@ifpackageloaded{caption}{}{\usepackage{caption}}
\AtBeginDocument{%
\ifdefined\contentsname
  \renewcommand*\contentsname{Table des matières}
\else
  \newcommand\contentsname{Table des matières}
\fi
\ifdefined\listfigurename
  \renewcommand*\listfigurename{Liste des Figures}
\else
  \newcommand\listfigurename{Liste des Figures}
\fi
\ifdefined\listtablename
  \renewcommand*\listtablename{Liste des Tables}
\else
  \newcommand\listtablename{Liste des Tables}
\fi
\ifdefined\figurename
  \renewcommand*\figurename{Figure}
\else
  \newcommand\figurename{Figure}
\fi
\ifdefined\tablename
  \renewcommand*\tablename{Table}
\else
  \newcommand\tablename{Table}
\fi
}
\@ifpackageloaded{float}{}{\usepackage{float}}
\floatstyle{ruled}
\@ifundefined{c@chapter}{\newfloat{codelisting}{h}{lop}}{\newfloat{codelisting}{h}{lop}[chapter]}
\floatname{codelisting}{Listing}
\newcommand*\listoflistings{\listof{codelisting}{Liste des Listings}}
\makeatother
\makeatletter
\makeatother
\makeatletter
\@ifpackageloaded{caption}{}{\usepackage{caption}}
\@ifpackageloaded{subcaption}{}{\usepackage{subcaption}}
\makeatother

\ifLuaTeX
\usepackage[bidi=basic]{babel}
\else
\usepackage[bidi=default]{babel}
\fi
\babelprovide[main,import]{french}
% get rid of language-specific shorthands (see #6817):
\let\LanguageShortHands\languageshorthands
\def\languageshorthands#1{}
\usepackage{bookmark}

\IfFileExists{xurl.sty}{\usepackage{xurl}}{} % add URL line breaks if available
\urlstyle{same} % disable monospaced font for URLs
\hypersetup{
  pdflang={fr},
  colorlinks=true,
  linkcolor={blue},
  filecolor={Maroon},
  citecolor={Blue},
  urlcolor={Blue},
  pdfcreator={LaTeX via pandoc}}


\author{}
\date{}

\begin{document}


\subsection{Questions fréquemment posées
(FAQ)}\label{questions-fruxe9quemment-posuxe9es-faq}

\subsubsection{Questions générales :}\label{questions-guxe9nuxe9rales}

\begin{enumerate}
\def\labelenumi{\arabic{enumi}.}
\tightlist
\item
  \textbf{Quel est l'emploi du temps des cours et les heures de bureau
  ?}

  \begin{itemize}
  \tightlist
  \item
    Les cours ont lieu tous les {[}lundi au jeudi{]} de 10h00 à 11h30
    HNE et de 20h00 à 21h30 HNE.
  \end{itemize}
\item
  \textbf{Comment puis-je vous contacter en dehors des heures de cours
  ?}

  \begin{itemize}
  \tightlist
  \item
    Vous pouvez nous contacter à
    \href{mailto:frenchwithkunal@gmail.com}{\nolinkurl{frenchwithkunal@gmail.com}}
    et on vous répondra dès que possible.
  \end{itemize}
\end{enumerate}

\subsubsection{Vocabulaire :}\label{vocabulaire}

\begin{enumerate}
\def\labelenumi{\arabic{enumi}.}
\setcounter{enumi}{2}
\tightlist
\item
  \textbf{Comment puis-je développer efficacement mon vocabulaire en
  français ?}

  \begin{itemize}
  \tightlist
  \item
    La pratique régulière, l'utilisation de cartes mémoire, la lecture
    de textes en français et la participation à des conversations avec
    des locuteurs natifs sont des moyens efficaces pour développer votre
    vocabulaire.
  \end{itemize}
\end{enumerate}

\subsubsection{Prononciation :}\label{prononciation}

\begin{enumerate}
\def\labelenumi{\arabic{enumi}.}
\setcounter{enumi}{3}
\tightlist
\item
  \textbf{Comment prononcer correctement les sons nasaux en français ?}

  \begin{itemize}
  \tightlist
  \item
    Les sons nasaux en français incluent ``an,'' ``en,'' ``on,'' et
    ``un.'' Ces sons sont produits en laissant l'air s'échapper par le
    nez. Pratiquez en écoutant des locuteurs natifs et en les imitant.
  \end{itemize}
\item
  \textbf{Quel est le meilleur moyen d'améliorer mon accent français ?}

  \begin{itemize}
  \tightlist
  \item
    Écoutez des locuteurs natifs, imitez leur prononciation et pratiquez
    régulièrement. S'enregistrer en parlant peut être d'une grande aide
    !
  \end{itemize}
\end{enumerate}

\subsubsection{Écoute et expression orale
:}\label{uxe9coute-et-expression-orale}

\begin{enumerate}
\def\labelenumi{\arabic{enumi}.}
\setcounter{enumi}{5}
\tightlist
\item
  \textbf{Comment puis-je améliorer mes compétences de compréhension
  orale en français ?}

  \begin{itemize}
  \tightlist
  \item
    Écoutez des podcasts en français, regardez des films ou des
    émissions de télévision en français, et essayez de vous immerger
    autant que possible dans la langue.
  \end{itemize}
\item
  \textbf{Quels sont des moyens efficaces pour pratiquer la langue
  parlée en français ?}

  \begin{itemize}
  \tightlist
  \item
    Pratiquez en parlant avec des locuteurs natifs, rejoignez un groupe
    d'échange linguistique, ou utilisez des applications d'apprentissage
    des langues qui se concentrent sur l'expression orale.
  \end{itemize}
\item
  \textbf{Y a-t-il des podcasts ou des vidéos que vous recommandez pour
  apprendre le français ?}

  \begin{itemize}
  \tightlist
  \item
    Quelques podcasts populaires incluent ``Coffee Break French,''
    ``InnerFrench,'' et ``FrenchPod101.'' Pour les vidéos, les chaînes
    YouTube comme ``Learn French with Alexa'' et ``Francais
    Authentique'' sont d'excellentes ressources.
  \end{itemize}
\end{enumerate}

\subsubsection{Lecture et écriture :}\label{lecture-et-uxe9criture}

\begin{enumerate}
\def\labelenumi{\arabic{enumi}.}
\setcounter{enumi}{8}
\tightlist
\item
  \textbf{Comment puis-je améliorer ma compréhension de lecture en
  français ?}

  \begin{itemize}
  \tightlist
  \item
    Commencez par des textes simples et passez progressivement à des
    textes plus complexes. Lisez des livres, des journaux et des
    articles en français, et essayez de comprendre le contexte.
  \end{itemize}
\item
  \textbf{Quels sont des bons livres ou articles en français pour les
  débutants ?}

  \begin{itemize}
  \tightlist
  \item
    Quelques bonnes options incluent ``Le Petit Prince'' d'Antoine de
    Saint-Exupéry, la série ``Le Petit Nicolas'' de René Goscinny, et
    des sites de nouvelles en français simples comme ``Le Monde des
    ados.''
  \end{itemize}
\item
  \textbf{Quelles sont des stratégies pour écrire des essais en français
  ?}

  \begin{itemize}
  \tightlist
  \item
    Planifiez la structure de votre essai, utilisez une variété de
    structures de phrases et pratiquez régulièrement l'écriture.
    Concentrez-vous sur la clarté et la cohérence, et relisez votre
    travail pour vérifier la grammaire et le vocabulaire.
  \end{itemize}
\end{enumerate}

\subsubsection{Aperçus culturels :}\label{aperuxe7us-culturels}

\begin{enumerate}
\def\labelenumi{\arabic{enumi}.}
\setcounter{enumi}{11}
\tightlist
\item
  \textbf{Quels sont des conseils culturels pour interagir avec des
  locuteurs français ?}

  \begin{itemize}
  \tightlist
  \item
    Soyez poli, utilisez des titres formels (Monsieur, Madame) lorsque
    c'est approprié, et apprenez quelques bases de l'étiquette, comme le
    salut par poignée de main ou par bise.
  \end{itemize}
\item
  \textbf{Pouvez-vous recommander des films ou des émissions de
  télévision en français à regarder ?}

  \begin{itemize}
  \tightlist
  \item
    Quelques films français populaires incluent ``Amélie,'' ``La
    Haine,'' et ``Intouchables.'' Pour les émissions de télévision,
    ``Call My Agent!'' et ``Lupin'' sont vivement recommandés.
  \end{itemize}
\end{enumerate}

\subsubsection{Ressources et outils :}\label{ressources-et-outils}

\begin{enumerate}
\def\labelenumi{\arabic{enumi}.}
\setcounter{enumi}{13}
\tightlist
\item
  \textbf{Quels dictionnaires ou livres de référence devrais-je utiliser
  ?}

  \begin{itemize}
  \tightlist
  \item
    ``Le Petit Robert'' et ``Larousse'' sont des dictionnaires français
    très réputés. Pour les livres de référence,
    ``\href{https://bpb-us-e1.wpmucdn.com/blogs.uoregon.edu/dist/f/8175/files/2014/06/Kendris_501_French_Verbs_Barrons-13u1653.pdf}{501
    French Verbs}'' de Christopher Kendris est un excellent outil pour
    les conjugaisons de verbes.
  \end{itemize}
\end{enumerate}

\subsubsection{Durée d'apprentissage :}\label{duruxe9e-dapprentissage}

\begin{enumerate}
\def\labelenumi{\arabic{enumi}.}
\setcounter{enumi}{14}
\tightlist
\item
  \textbf{Combien de temps faut-il pour apprendre le français ?}

  \begin{itemize}
  \tightlist
  \item
    Avec une guidance adéquate et de la pratique (environ 3 heures par
    jour), il faut généralement environ 6 mois pour atteindre un niveau
    intermédiaire (B2/CLB5) en français. Un effort constant et une
    immersion dans la langue peuvent accélérer considérablement le
    processus d'apprentissage.
  \end{itemize}
\end{enumerate}




\end{document}
