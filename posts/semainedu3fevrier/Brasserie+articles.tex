% Options for packages loaded elsewhere
\PassOptionsToPackage{unicode}{hyperref}
\PassOptionsToPackage{hyphens}{url}
\PassOptionsToPackage{dvipsnames,svgnames,x11names}{xcolor}
%
\documentclass[
  16pt,
  letterpaper,
  DIV=11,
  numbers=noendperiod]{scrartcl}

\usepackage{amsmath,amssymb}
\usepackage{iftex}
\ifPDFTeX
  \usepackage[T1]{fontenc}
  \usepackage[utf8]{inputenc}
  \usepackage{textcomp} % provide euro and other symbols
\else % if luatex or xetex
  \usepackage{unicode-math}
  \defaultfontfeatures{Scale=MatchLowercase}
  \defaultfontfeatures[\rmfamily]{Ligatures=TeX,Scale=1}
\fi
\usepackage{lmodern}
\ifPDFTeX\else  
    % xetex/luatex font selection
\fi
% Use upquote if available, for straight quotes in verbatim environments
\IfFileExists{upquote.sty}{\usepackage{upquote}}{}
\IfFileExists{microtype.sty}{% use microtype if available
  \usepackage[]{microtype}
  \UseMicrotypeSet[protrusion]{basicmath} % disable protrusion for tt fonts
}{}
\makeatletter
\@ifundefined{KOMAClassName}{% if non-KOMA class
  \IfFileExists{parskip.sty}{%
    \usepackage{parskip}
  }{% else
    \setlength{\parindent}{0pt}
    \setlength{\parskip}{6pt plus 2pt minus 1pt}}
}{% if KOMA class
  \KOMAoptions{parskip=half}}
\makeatother
\usepackage{xcolor}
\setlength{\emergencystretch}{3em} % prevent overfull lines
\setcounter{secnumdepth}{-\maxdimen} % remove section numbering
% Make \paragraph and \subparagraph free-standing
\makeatletter
\ifx\paragraph\undefined\else
  \let\oldparagraph\paragraph
  \renewcommand{\paragraph}{
    \@ifstar
      \xxxParagraphStar
      \xxxParagraphNoStar
  }
  \newcommand{\xxxParagraphStar}[1]{\oldparagraph*{#1}\mbox{}}
  \newcommand{\xxxParagraphNoStar}[1]{\oldparagraph{#1}\mbox{}}
\fi
\ifx\subparagraph\undefined\else
  \let\oldsubparagraph\subparagraph
  \renewcommand{\subparagraph}{
    \@ifstar
      \xxxSubParagraphStar
      \xxxSubParagraphNoStar
  }
  \newcommand{\xxxSubParagraphStar}[1]{\oldsubparagraph*{#1}\mbox{}}
  \newcommand{\xxxSubParagraphNoStar}[1]{\oldsubparagraph{#1}\mbox{}}
\fi
\makeatother


\providecommand{\tightlist}{%
  \setlength{\itemsep}{0pt}\setlength{\parskip}{0pt}}\usepackage{longtable,booktabs,array}
\usepackage{calc} % for calculating minipage widths
% Correct order of tables after \paragraph or \subparagraph
\usepackage{etoolbox}
\makeatletter
\patchcmd\longtable{\par}{\if@noskipsec\mbox{}\fi\par}{}{}
\makeatother
% Allow footnotes in longtable head/foot
\IfFileExists{footnotehyper.sty}{\usepackage{footnotehyper}}{\usepackage{footnote}}
\makesavenoteenv{longtable}
\usepackage{graphicx}
\makeatletter
\newsavebox\pandoc@box
\newcommand*\pandocbounded[1]{% scales image to fit in text height/width
  \sbox\pandoc@box{#1}%
  \Gscale@div\@tempa{\textheight}{\dimexpr\ht\pandoc@box+\dp\pandoc@box\relax}%
  \Gscale@div\@tempb{\linewidth}{\wd\pandoc@box}%
  \ifdim\@tempb\p@<\@tempa\p@\let\@tempa\@tempb\fi% select the smaller of both
  \ifdim\@tempa\p@<\p@\scalebox{\@tempa}{\usebox\pandoc@box}%
  \else\usebox{\pandoc@box}%
  \fi%
}
% Set default figure placement to htbp
\def\fps@figure{htbp}
\makeatother

\KOMAoption{captions}{tableheading}
\makeatletter
\@ifpackageloaded{caption}{}{\usepackage{caption}}
\AtBeginDocument{%
\ifdefined\contentsname
  \renewcommand*\contentsname{Table of contents}
\else
  \newcommand\contentsname{Table of contents}
\fi
\ifdefined\listfigurename
  \renewcommand*\listfigurename{List of Figures}
\else
  \newcommand\listfigurename{List of Figures}
\fi
\ifdefined\listtablename
  \renewcommand*\listtablename{List of Tables}
\else
  \newcommand\listtablename{List of Tables}
\fi
\ifdefined\figurename
  \renewcommand*\figurename{Figure}
\else
  \newcommand\figurename{Figure}
\fi
\ifdefined\tablename
  \renewcommand*\tablename{Table}
\else
  \newcommand\tablename{Table}
\fi
}
\@ifpackageloaded{float}{}{\usepackage{float}}
\floatstyle{ruled}
\@ifundefined{c@chapter}{\newfloat{codelisting}{h}{lop}}{\newfloat{codelisting}{h}{lop}[chapter]}
\floatname{codelisting}{Listing}
\newcommand*\listoflistings{\listof{codelisting}{List of Listings}}
\makeatother
\makeatletter
\makeatother
\makeatletter
\@ifpackageloaded{caption}{}{\usepackage{caption}}
\@ifpackageloaded{subcaption}{}{\usepackage{subcaption}}
\makeatother

\ifLuaTeX
\usepackage[bidi=basic]{babel}
\else
\usepackage[bidi=default]{babel}
\fi
\babelprovide[main,import]{english}
% get rid of language-specific shorthands (see #6817):
\let\LanguageShortHands\languageshorthands
\def\languageshorthands#1{}
\ifLuaTeX
  \usepackage[english]{selnolig} % disable illegal ligatures
\fi
\usepackage{bookmark}

\IfFileExists{xurl.sty}{\usepackage{xurl}}{} % add URL line breaks if available
\urlstyle{same} % disable monospaced font for URLs
\hypersetup{
  pdftitle={Brasserie+articles},
  pdfauthor={Kunal Khurana},
  pdflang={en},
  colorlinks=true,
  linkcolor={blue},
  filecolor={Maroon},
  citecolor={Blue},
  urlcolor={Blue},
  pdfcreator={LaTeX via pandoc}}


\title{Brasserie+articles}
\author{Kunal Khurana}
\date{2025-02-07}

\begin{document}
\maketitle


\phantomsection\label{copy-message}{Link copied!}

\phantomsection\label{react-root}

In today's
\href{https://drive.google.com/file/d/1ESQ5L5H1BAVH96tB-sZq4b_lJrkGgsdO/view?usp=drive_link}{lecture},
we focused on some practical and everyday scenarios that are super
helpful for navigating life in France: \textbf{ordering drinks in a
café} and \textbf{buying refreshments in a brasserie}. We also touched
on the use of \textbf{articles} (like \emph{le, la, les, un, une, des})
in French, which are essential for constructing sentences correctly.

\subsubsection{Ordering Drinks in a
Café}\label{ordering-drinks-in-a-cafuxe9}

Ordering in a café is a quintessential French experience, and we learned
how to do it confidently. For example:\\
- \textbf{« Je voudrais un café, s'il vous plaît. »} (I would like a
coffee, please.)\\
- \textbf{« Un thé vert, s'il vous plaît. »} (A green tea, please.)\\
- \textbf{« L'addition, s'il vous plaît. »} (The bill, please.)

These simple phrases can make your café visits smooth and enjoyable.

\subsubsection{Buying Refreshments in a
Brasserie}\label{buying-refreshments-in-a-brasserie}

A brasserie is a great place to grab a quick snack or drink, and we
practiced how to order there too. For instance:\\
- \textbf{« Je prends une bouteille d'eau, s'il vous plaît. »} (I'll
take a bottle of water, please.)\\
- \textbf{« Une limonade, s'il vous plaît. »} (A lemonade, please.)\\
- \textbf{« Combien ça coûte ? »} (How much does it cost?)

These phrases are perfect for when you're out and about and need a quick
refreshment.

\subsubsection{Articles in French}\label{articles-in-french}

We also discussed \textbf{articles} (\emph{le, la, les, un, une, des}),
which are small but mighty words that define nouns. For example:\\
- \textbf{« Le café »} (the coffee) vs.~\textbf{« Un café »} (a
coffee).\\
- \textbf{« La limonade »} (the lemonade) vs.~\textbf{« Une limonade »}
(a lemonade).

One helpful tip we learned is that \textbf{definite articles (le, la,
les)} are not used after certain prepositions and verbs, such as
\textbf{parler, de, and en}. For example:\\
- \textbf{« Je parle français. »} (I speak French.) -- No article after
\emph{parler}.\\
- \textbf{« J'ai besoin de café. »} (I need coffee.) -- No article after
\emph{de}.\\
- \textbf{« C'est en anglais. »} (It's in English.) -- No article after
\emph{en}.

This rule simplifies learning because it reduces the need to memorize
articles in these specific contexts.

This lecture was packed with practical tips that are perfect for anyone
looking to feel more confident in French-speaking environments. Whether
you're sipping coffee at a café or grabbing a drink at a brasserie,
these phrases will definitely come in handy. À bientôt !




\end{document}
